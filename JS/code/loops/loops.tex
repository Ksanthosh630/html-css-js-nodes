
loops: loops are used to iterate certain statements. Basically loops are two types
  i. Entry control loops
  ii. Exit Control loops

i. Entry Control loops:
    These loops checks the condition at the starting of loop. it will enter only if the condition is true and exit when condition is false. until the condition is false the loop iterates
    eg: while, for
      check condition
         true
           runs statement1
           again check condition
     false
         stop
ii. Exit Control loops:
    These loops checks the condition at the ending of loop. it will enter loop without condition but while exiting it checks the condition, if it is true it re runs again , else it will stop
   eg: do while

   i. while: while is entry control loop that iterates until the condition fails.
    syntax: initialization;
            while(condition){
                //statements
                updation;
            }

      eg: let i=1;
      while(i<10){
        console.log(i);
        i++;
      }

    Mostly when initialization, condition and updation are need to be in separate places, then will go for while

ii. do while : do while is a exit control loop which will get initial run without check and will iterate until condition false.
     initialization;
      do{
        //statements;
        updation;
      }while(condition);

    task : print even number from 10 to 1 in reverse order using while and dowhile
    output: 10
            8
            6
            4
            2
    task2 : print all letters of a string in reverse order without space
    sample input: "I love JS"
    sample output:"SJevolI"

let s="I love Js";
let s1='';
let i =s.length-1;
while(i>=0){
if (s.charAt(i)!= ' ')
{
s1=s1.concat(s.charAt(i));
}
i--;
}

task 3:
print in console
*
* *
* * *
* * * *
task3 solution:
-----------------
function f1(){
    console.log('with space printing * ');

    const n=4;
    const s1="*"
    var s3="";

    for(let i=1;i<=n;i++)
         {
            var s="";
            for(let j=1;j<=i;j++)
            {
                s=s.concat( s1+" "+" ")
            }
           // console.log(s);
           
            s3=s3.concat(s+"\n")
            
         }
         console.log(s3);
}
f1();



iii. for loop:  for loop is used as a entry control loop but initialization, condition and updation are in same line

for(initialization;condition;updation){
  //statements
}

//print numbers from 10 to 1 using for

for (let i = 10; i >= 1; i--)
{
    console.log(i)
}

Variations of for:
  In js, For loop have so many Variations such as for..of, for..in and forEach
for..of: it is used to print all values from the array
syntax: for(let Variable of array){
  console.log(Variable)
}
here Variable is considered as Element 

for..in : it is used to print all values of object using key values
syntax : for(let Variable in array/object){
  console.log(Variable)
}

here Variable is considered as index or object key